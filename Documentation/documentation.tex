\documentclass[12pt]{article}

\usepackage{sbc-template}

\usepackage{graphicx,url}

\usepackage[utf8]{inputenc}  

     
\sloppy

\title{Instructions for Authors of SBC Conferences\\ Papers and Abstracts}

\author{Turcan Olga, Gurdis Viorel}

\address{Babes-Bolyai University, Cluj-Napoca}


\begin{document} 

\maketitle

\begin{abstract}
  Living in a big city, it's a common problem to find available parking 
  spaces for your car. Having a real time parking space 
  detection system based on images provided by CCTVs could considerably 
  improve the parking experience at a relatively low cost.
\end{abstract}


\section{Introduction}

Statistically the average time spent searching for a parking spot represents \textbf{7.8 minutes}.
That is a waste of time and can also cause traffic congestion. There are several types of 
parking spaces monitoring systems, including counter-based and sensor-based, which try to solve the mentioned problem.
Counter-based systems work by counting cars at the parking lot entrance, but their disadvantage is that they don't provide 
concrete informations about available spots, leaving the task of finding the place to the driver.
A more precise solution could be sensor-based systems which provide availability information about each 
spot specifically. However, this solution implies higher costs (c.a \textbf{\$40} per unit). 
An alternative approach would be using video provided by already installed surveillance cameras for real-time 
detection by implementing a computer vision system.
This method relies on the detection of vehicles in delimited space areas and classification of a spot as 
available or occupied. The processed information will be delivered by a server in form of a web page and updated 
in real-time, which would make the solution accessible for everyone. That would definetely save time of the driver 
and considerably reduce traffic problems.
Challenges that we should expect will be related to the scalability of the application. We should, for example, take into 
consideration that the application should work in different weather conditions. We also focus on the accessibility 
of the processed information in real-time, for which we will try to compare different AI algorithms and will choose 
the one with the best balance between accurate results and small processing time.

\section{Related Work} 

One approach to video-based real-time parking space detection application 
was described in the academic paper of \cite{tschentscher}. To minimize 
weather and lighting conditions influences and to maximize the accuracy of the 
results, the approach evaluates several combinations of feature extractors and 
machine learning algorithms. They used a self-built dataset containing 
ca. 10,000 samples, from which they extracted features like 
Color Histograms [RGB, HSV, YUV] (to distinguish between asphalt color and the 
cars, to solve problems with brightness), Gradient Histograms, 
Difference-of-Gaussian Histograms and Haar-like (to extract edge information). 
Three classifiers were trained and compared to each other based on features 
mentioned above: k-Nearest Neighbor, Linear Discriminant Analysis, Support 
Vector Machine. The final solution relies on HSV color histogram and 
Difference-of-Gaussian features and a SVM classifier, which reached an 
accuracy of 99.8 \%.
\\
\\
An alternate approach for vacant parking space detection is described in 
\cite{debaditya}, which uses features extracted by a pre-trained CNN to 
train an SVM classifier for the detection of parking occupancy in a 
CCTV image sequence.
The CNN extracts features from the publicly available PKLot dataset which 
consists of more than 12000 images collected from 3 parking sites on 
different weather conditions. 
Consequently, the extracted features from images of the PKLot dataset 
were used to train and test a binary SVM classifier.
The evaluation of the classification accuracy is done by cross validation 
on the PKLot dataset and the transfer learning ability on the Barry 
Street dataset, which was created for the purpose of that research and 
includes a sequence of images captured by a camera overlooking a street 
with marked parking places.
\\
\\
The binary classification using the deep features achieved consistently 
reliable results with an average accuracyof 99.7\% across different weather 
conditions for the PKLot dataset.
As transfer learning was a more challenging task because the classifier 
was required to recognise unfamiliar images, the classification of Barry 
street images achieved the overall accuracy of 96.65\%.
It is worth mentioning that the processing time for each image segment 
of the parking spaces is 0.067 seconds on a simple desktop computer 
which means it takes approximately 2 seconds to process all the parking 
spaces in an image and hence the solution is suitable for real-time 
applications without any dedicated hardware.

\subsection{Useful Tools}
The following tools were used for implementing above mentioned approaches.
\begin{itemize}
  \item Tensorflow
  \item Keras
  \item scikit-learn
\end{itemize}

\bibliographystyle{sbc}
\bibliography{sbc-template}

\end{document}
